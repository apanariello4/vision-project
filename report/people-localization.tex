\section{People Localization}
Our approach to achieve this task is based on the quality of the detection and painting retrieval. In fact, if our trained model correctly detect a person, in order to localize that person, we use informations about the paintings detected by the model and retrieved from the database.

\subsection{Evaluation}
We have access to the paintings\_db, with informations about the room where each painting is located, but there are two main problems that worsen our evaluation:

\begin{enumerate}[label=\alph*)]
	\item \label{retrieval_case} The painting retrieval almost performs a random choice when a painting is in the database.
    \item \label{painting_other_room} We do not take into account the scenario in which the camera and the person are in a room, while a detected painting is in another room, visible through a door.
\end{enumerate}

In our evaluation we discarded all cases that match \ref*{painting_other_room} and the results are all depending on the painting retrieval, which are show in table \ref*{tab:retrieval_eval}.