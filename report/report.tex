\documentclass[11pt,oneside,a4paper,twocolumn]{article}
\usepackage{booktabs}
\usepackage{hyperref}
\usepackage[utf8]{inputenc}
\usepackage[english]{babel}
\usepackage{fancyhdr}
\usepackage{graphicx}
\usepackage{parskip}
\usepackage[affil-it]{authblk}
\usepackage{lipsum}
\usepackage[sorting=none]{biblatex}
\usepackage{csquotes}
\usepackage{multirow}
\usepackage[table,xcdraw]{xcolor}
\usepackage{rotating}
\usepackage{enumitem}
\usepackage{subcaption}
\usepackage{caption}

\setlength{\headheight}{14pt}


\addbibresource{bibliography.bib}



\makeatletter
\def\thm@space@setup{%
  \thm@preskip=\parskip \thm@postskip=0pt
}

\pagestyle{fancy}

\title{Painting Detection and Rectification with People Detection}

\author{Aniello Panariello - 140195}
\author{Fabrizio Sorgente - 133855}
\author{Emanuele Fenocchi - 148869}
\affil{University of Modena and Reggio Emilia}

\begin{document}

\maketitle

\section{Introduction}
The aim of this work is to detect paintings and people inside a museum environment and then perform retrieval and rectification of the detected painting from a database of high quality images, we also detect statues. The detection of the three objects is performed with a custom trained YOLO network, while the retrieval is done by ORB keypoints. For the rectification we exploit the keypoints obtained by the ORB to find the homography matrix. Once we have found the paintings and the people we can localize the latter by getting the localization of the painting. The direction in which the person is facing is computed by a face detection and assuming that if the person is not looking at the camera then he is facing a painting.
% We also process screenshots from the 3D model provided substituting paintings with the hig quality images using an inverse approach to that used for rectification.

\section{Related Works}
Related Works

\section{Approach}
In this section we will describe the main steps of the pipeline with their evaluation.
\section{Painting Detection}
\lipsum[1-4]

\section{Painting Retrieval}
Painting retrieval uses ORB~\cite{orb} keypoints detector and descriptor to find matches between two images. ORB, aka "Oriented FAST and Rotated BRIEF", is at two orders of magnitude faster than the old used SIFT~\cite{sift} and this is the main reason why we choose to use this method in order to achieve the painting retrieval with a good performance/results ratio.
\subsection{Painting Rectification}
We manage to perform a very accurate rectification when the painting retrieval is able to correctly fetch the painting entry from our database. In this case, not only we have the correct correspondence of paintings, but we can also exploit the descriptors obtained from ORB to compute an extremely accurate homography matrix, thanks to the large number of keypoints, to then compute the projective transformation.

\begin{figure}[h!]
    \centering
    \includegraphics[width=.49\textwidth]{pictures/painting_rectification/rectification.png}
    \caption{example of painting rectification}
    \label{fig:rectification_ex}
\end{figure}

When the retrieval fails, for the lack of descriptors or when the painting is missing from the database, we use Harris Corner Detetor~\cite{harris-corner} to find the corners of the painting and use them as source points for our homography matrix. This method has a worse performance than the descriptors method, since the corners are not always precise.

\subsection{People Localization}
Our approach to achieve this task is based on the quality of the detection and painting retrieval. In fact, if our trained model correctly detect a person, in order to localize that person, we use information about the paintings detected by the model and retrieved from the database.

\subsubsection{Evaluation}
We have access to the paintings\_db, with informations about the room where each painting is located, but there are two main problems that worsen our evaluation:

\begin{enumerate}[label=\alph*)]
    \item \label{retrieval_case} The painting retrieval finds a match only in 50\% of the cases.
    \item \label{painting_other_room} We do not take into account the scenario in which the camera and the person are in a room, while a detected painting is in another room, visible through a door.
\end{enumerate}

In our evaluation we discarded all cases that match \ref*{painting_other_room} and the results are all depending on the painting retrieval, which are shown in table \ref*{tab:retrieval_eval}.

in the first case we can do nothing

\section{Face detection}
We used Haar Cascade classifiers \cite{haar_cascade} proposed by Paul Viola and Michael Jones in order to achieve this task. First, we use our trained model to detect a person, then the ROI of that person is passed to the face detector. The face detector performs the algorithm to check if a person is facing a painting, based on these following scenarios:
\begin{enumerate}[label=\alph*)]
    \item Face found
    \begin{enumerate}[label=(\roman*)]
        \item \label{Eyes found} Eyes found (at least one)
        \item \label{Eyes not found} Eyes not found
    \end{enumerate}
    \item \label{Face not found} Face not found
\end{enumerate}

\begin{figure}[h!]
    \centering
    \includegraphics[width=0.2\textwidth]{pictures/face_detection/face_det2}
    \caption{the case in which the face detector correctly found a face from the person ROI}
    \label{fig:Eyes}
\end{figure}

In case \ref*{Eyes found}, if the face is found with its eyes, we assume that the person is facing the camera therefore he is not facing a painting, like in fig. \ref{fig:Eyes}.
\begin{figure}[h!]
    \centering
    \includegraphics[width=0.3\textwidth]{pictures/face_detection/face_det1}
    \caption{the case where the face detector can't find the face}
    \label{fig:No_eyes}
\end{figure}
In case \ref*{Eyes not found}, if the face is found but the eyes are not detected, we assume that the person is facing a painting, this case is a particular scenario where the person could be in profile. The case \ref*{Face not found} has the same assumptions of the case \ref*{Eyes not found} because if the face is not detected, this could mean that the person is turning his back to the camera and, therefore, is possibly facing a painting, like in fig \ref{fig:No_eyes}. In these last scenarios, we take into account the paintings ROIs and we check if the person ROI overlaps a painting ROI, in this case the person is in front of a painting.

\subsection{Evaluation}
The assumptions we've made, have allowed us to model most of the possible cases, but since the videos where there is a clearly visible person are only few, the test evaluation may not represent the real accuracy of our approach. As a result, the test has been done using 2 frames per second for each videos where there is at least one person for more than 3 seconds, with a total of 8 videos. 466 are the total frames used and only 121 are the optimal candidates for the test, where a person is clearly visible, and was not too far from the camera. Since the face detector takes in input the person and paintings ROIs, our test is also affected by the quality of our trained model, therefore, not detecting some of the people and paintings, led us to discard another 42 frames.

After all of these reductions, we analyzed a total of 84 frames with 94 people inside it. The evaluation results are shown in table \ref{tab:face_detection_eval} and gave us an accuracy of: \[Accuracy = \frac{14+57}{94} \approx 0.85\]

\begin{table}[ht]
    \centering
    \begin{tabular}{cccc}
        & & \multicolumn{2}{c}{\textit{Facing painting}}                                                      \\ \cline{3-4}
        & \multicolumn{1}{c|}{}              & \multicolumn{1}{c|}{\textbf{True}}              & \multicolumn{1}{c|}{\textbf{False}}             \\ \cline{2-4}
        \multicolumn{1}{c|}{} & \multicolumn{1}{c|}{\textbf{True}} & \multicolumn{1}{c|}{14} & \multicolumn{1}{c|}{10} \\ \cline{2-4}
        \multicolumn{1}{c|}{\multirow{-2}{*}{\textit{\begin{tabular}[c]{@{}c@{}}
                                                         Face detector\\ answer
        \end{tabular}}}} &
        \multicolumn{1}{c|}{\textbf{False}} &
        \multicolumn{1}{c|}{13} &
        \multicolumn{1}{c|}{57} \\ \cline{2-4}
    \end{tabular}%
    \caption{Face detection results}
    \label{tab:face_detection_eval}
\end{table}

\subsection{3D Rectification}


\begin{figure*}[h]
    \minipage{0.45\textwidth}
      \includegraphics[width=\linewidth]{pictures/painting_detection/3d_rectification_original.PNG}
      \caption*{Original 3D model painting}\label{fig:rectification_original}
    \endminipage\hfill
    \minipage{0.45\textwidth}
      \includegraphics[width=\linewidth]{pictures/painting_detection/3d_rectification_warped.PNG}
      \caption*{Replaced painting}\label{fig:rectification_warped}
    \endminipage\hfill
    \caption{Rectification painting of 3D model}\label{fig:3d-warping}
\end{figure*}


The 3D Rectification is the last optional task that has been done. The scope was to replace the painting that was present in a 3D model with the same painting but with an higher resolution. Have been taken some screenshot of the 3D model with an off-line approach, then were extracted all the ROI that were present in the acquired image and each painting of the model was confronted with all the painting in our database. The painting identified was first warped with an homography trasformation in order to be in the same prospective of the painting in the screenshot and then was replaced with the previous one; in case of mismatch nothing was done. One of the results that has been achieved is showed in the fig.~\ref{fig:3d-warping}.

\section{Results}
In this work we were able to train a neural network with high performance to detect paintings, people and statues with a $mAP$ of $90.45\%$. This is a great result for a custom network and a not so large dataset. The neural network detection lays the foundation for all the subsequent steps of the pipeline, having a good performance in the first step is the key to have a good performance also in other steps.

The painting retrieval is our bottleneck since it is able to correctly retrieve paintings just in the $52\%$ of the cases, this is also due to the fact that not every painting is in the database and the bad luminance and projective condition of some frames.

Regarding the painting rectification and people localization, these performance are strictly tied to the previous step since, when the retrieval is correct, this step works without flaws.

The face detection, which is performed in order to determine if a person is facing a painting or not, exploits the detection of a person, this means that it has an $AP$ of $75.45\%$ as we can see in tab.~\ref{tab:detection_performance}. On these detections we are able to verify if a person is facing a painting with an accuracy of $85\%$, as we can see in tab.~\ref{tab:face_detection_eval}, using a classic computer vision algorithm without the use of neural networks.

In the last step, the rectification of paintings of the 3D model is performed, using the same pipeline as before with a change in the rectification step. The performance of this step are the same as the rectification in 2D frames.

\section{Discussion}
In this project we compared the performance of classic computer vision algorithms and neural networks. The neural network we have trained, greatly outperforms our implementation of CCL~\cite{Grana_ccl} or thresholding algorithms. While these algorithms could have been optimized with some parameters tweaking, the deep learning approach is faster to implement, once a dataset is available, and is able to generalize much better.

In all the steps, except the detection step, classical algorithms were used, such as descriptors, projective transformation, face detector and so on. Not using neural networks in these steps, allowed us to have better computing performance and a faster pipeline, this would not have been possible if we only used deep learning.

Moreover, in steps such as painting retrieval, the use of a neural network would have been impractical, since it would have been impossible to obtain an exhaustive dataset of paintings, we only had one image per painting.

In conclusion, the combined use of both classical and deep learning algorithms is inevitable to achieve optimal performance and to solve some tasks that otherwise would have been impossible.


\printbibliography[heading=bibintoc,title={Bibliography}]\thispagestyle{empty}



% \twocolumn[
%   \begin{@twocolumnfalse}
%     \section*{Additional Resources}

\begin{figure*}[ht!]
  \centering
      \begin{subfigure}[t]{0.3\textwidth}
        \includegraphics[width=\linewidth]{pictures/painting_detection/Frame.png}
        \caption*{Frame from video}\label{fig:Frame}
      \end{subfigure}\hfill
      \begin{subfigure}[t]{0.3\textwidth}
        \includegraphics[width=\linewidth]{pictures/painting_detection/2-adaptive_threshold.PNG}
        \caption*{Adaptive threshold}\label{fig:adaptive_threshold}
      \end{subfigure}\hfill
      \begin{subfigure}[t]{0.3\textwidth}%
        \includegraphics[width=\linewidth]{pictures/painting_detection/3-median_blur.PNG}
        \caption*{Median blur}\label{fig:median_blur}
      \end{subfigure}
      \\
      \begin{subfigure}[b]{0.3\textwidth}
        \includegraphics[width=\linewidth]{pictures/painting_detection/4-opening.PNG}
        \caption*{Opening}\label{fig:opening}
      \end{subfigure}\hfill
      \begin{subfigure}[b]{0.3\textwidth}
      \captionsetup{font=small}
        \includegraphics[width=\linewidth]{pictures/painting_detection/5-ccl.PNG}
        \caption*{Connected Component Labeling}\label{fig:ccl}
      \end{subfigure}\hfill
      \begin{subfigure}[b]{0.3\textwidth}%
        \includegraphics[width=\linewidth]{pictures/painting_detection/6-bbox.PNG}
        \caption*{Painting Detection}\label{fig:bbox}
      \end{subfigure}
      \caption{Detection pipeline without neural network.} \label{fig:pipeline_detection}
  \end{figure*}

%\section*{Additional Resources}

% \begin{figure}[h!]
%   \centering
%   \resizebox{1\textwidth}{!}{
%     \centering
%     \resizebox{0.6\textwidth}{!}{
%       \minipage{0.3\textwidth}
%         \includegraphics[width=\linewidth]{pictures/painting_detection/Frame.png}
%         \caption*{Frame from video}\label{fig:Frame}
%       \endminipage\hfill
%       \minipage{0.3\textwidth}
%         \includegraphics[width=\linewidth]{pictures/painting_detection/2-adaptive_threshold.PNG}
%         \caption*{Adaptive threshold}\label{fig:adaptive_threshold}
%       \endminipage\hfill
%       \minipage{0.3\textwidth}%
%         \includegraphics[width=\linewidth]{pictures/painting_detection/3-median_blur.PNG}
%         \caption*{Median blur}\label{fig:median_blur}
%       \endminipage}
%       \centering
%       \resizebox{0.6\textwidth}{!}{
%       \minipage{0.3\textwidth}
%         \includegraphics[width=\linewidth]{pictures/painting_detection/4-opening.PNG}
%         \caption*{Opening}\label{fig:opening}
%       \endminipage\hfill
%       \minipage{0.3\textwidth}
%       \captionsetup{font=small}
%         \includegraphics[width=\linewidth]{pictures/painting_detection/5-ccl.PNG}
%         \caption*{Connected Component Labeling}\label{fig:ccl}
%       \endminipage\hfill
%       \minipage{0.3\textwidth}%
%         \includegraphics[width=\linewidth]{pictures/painting_detection/6-bbox.PNG}
%         \caption*{Painting Detection}\label{fig:bbox}
%       \endminipage}}
%       \captionsetup{justification=centering}
%       \caption{Detection pipeline without neural network.} \label{fig:pipeline_detection}
%   \end{figure}

%   \end{@twocolumnfalse}
% ]
\onecolumn
\thispagestyle{empty}
\section*{Additional Resources}

\begin{figure*}[ht!]
  \centering
      \begin{subfigure}[t]{0.3\textwidth}
        \includegraphics[width=\linewidth]{pictures/painting_detection/Frame.png}
        \caption*{Frame from video}\label{fig:Frame}
      \end{subfigure}\hfill
      \begin{subfigure}[t]{0.3\textwidth}
        \includegraphics[width=\linewidth]{pictures/painting_detection/2-adaptive_threshold.PNG}
        \caption*{Adaptive threshold}\label{fig:adaptive_threshold}
      \end{subfigure}\hfill
      \begin{subfigure}[t]{0.3\textwidth}%
        \includegraphics[width=\linewidth]{pictures/painting_detection/3-median_blur.PNG}
        \caption*{Median blur}\label{fig:median_blur}
      \end{subfigure}
      \\
      \begin{subfigure}[b]{0.3\textwidth}
        \includegraphics[width=\linewidth]{pictures/painting_detection/4-opening.PNG}
        \caption*{Opening}\label{fig:opening}
      \end{subfigure}\hfill
      \begin{subfigure}[b]{0.3\textwidth}
      \captionsetup{font=small}
        \includegraphics[width=\linewidth]{pictures/painting_detection/5-ccl.PNG}
        \caption*{Connected Component Labeling}\label{fig:ccl}
      \end{subfigure}\hfill
      \begin{subfigure}[b]{0.3\textwidth}%
        \includegraphics[width=\linewidth]{pictures/painting_detection/6-bbox.PNG}
        \caption*{Painting Detection}\label{fig:bbox}
      \end{subfigure}
      \caption{Detection pipeline without neural network.} \label{fig:pipeline_detection}
  \end{figure*}

%\section*{Additional Resources}

% \begin{figure}[h!]
%   \centering
%   \resizebox{1\textwidth}{!}{
%     \centering
%     \resizebox{0.6\textwidth}{!}{
%       \minipage{0.3\textwidth}
%         \includegraphics[width=\linewidth]{pictures/painting_detection/Frame.png}
%         \caption*{Frame from video}\label{fig:Frame}
%       \endminipage\hfill
%       \minipage{0.3\textwidth}
%         \includegraphics[width=\linewidth]{pictures/painting_detection/2-adaptive_threshold.PNG}
%         \caption*{Adaptive threshold}\label{fig:adaptive_threshold}
%       \endminipage\hfill
%       \minipage{0.3\textwidth}%
%         \includegraphics[width=\linewidth]{pictures/painting_detection/3-median_blur.PNG}
%         \caption*{Median blur}\label{fig:median_blur}
%       \endminipage}
%       \centering
%       \resizebox{0.6\textwidth}{!}{
%       \minipage{0.3\textwidth}
%         \includegraphics[width=\linewidth]{pictures/painting_detection/4-opening.PNG}
%         \caption*{Opening}\label{fig:opening}
%       \endminipage\hfill
%       \minipage{0.3\textwidth}
%       \captionsetup{font=small}
%         \includegraphics[width=\linewidth]{pictures/painting_detection/5-ccl.PNG}
%         \caption*{Connected Component Labeling}\label{fig:ccl}
%       \endminipage\hfill
%       \minipage{0.3\textwidth}%
%         \includegraphics[width=\linewidth]{pictures/painting_detection/6-bbox.PNG}
%         \caption*{Painting Detection}\label{fig:bbox}
%       \endminipage}}
%       \captionsetup{justification=centering}
%       \caption{Detection pipeline without neural network.} \label{fig:pipeline_detection}
%   \end{figure}

\end{document}
